
% Default to the notebook output style

    


% Inherit from the specified cell style.




    
\documentclass{article}

    
    
    \usepackage{graphicx} % Used to insert images
    \usepackage{adjustbox} % Used to constrain images to a maximum size 
    \usepackage{color} % Allow colors to be defined
    \usepackage{enumerate} % Needed for markdown enumerations to work
    \usepackage{geometry} % Used to adjust the document margins
    \usepackage{amsmath} % Equations
    \usepackage{amssymb} % Equations
    \usepackage[mathletters]{ucs} % Extended unicode (utf-8) support
    \usepackage[utf8x]{inputenc} % Allow utf-8 characters in the tex document
    \usepackage{fancyvrb} % verbatim replacement that allows latex
    \usepackage{grffile} % extends the file name processing of package graphics 
                         % to support a larger range 
    % The hyperref package gives us a pdf with properly built
    % internal navigation ('pdf bookmarks' for the table of contents,
    % internal cross-reference links, web links for URLs, etc.)
    \usepackage{hyperref}
    \usepackage{longtable} % longtable support required by pandoc >1.10
    \usepackage{booktabs}  % table support for pandoc > 1.12.2
    

    
    
    \definecolor{orange}{cmyk}{0,0.4,0.8,0.2}
    \definecolor{darkorange}{rgb}{.71,0.21,0.01}
    \definecolor{darkgreen}{rgb}{.12,.54,.11}
    \definecolor{myteal}{rgb}{.26, .44, .56}
    \definecolor{gray}{gray}{0.45}
    \definecolor{lightgray}{gray}{.95}
    \definecolor{mediumgray}{gray}{.8}
    \definecolor{inputbackground}{rgb}{.95, .95, .85}
    \definecolor{outputbackground}{rgb}{.95, .95, .95}
    \definecolor{traceback}{rgb}{1, .95, .95}
    % ansi colors
    \definecolor{red}{rgb}{.6,0,0}
    \definecolor{green}{rgb}{0,.65,0}
    \definecolor{brown}{rgb}{0.6,0.6,0}
    \definecolor{blue}{rgb}{0,.145,.698}
    \definecolor{purple}{rgb}{.698,.145,.698}
    \definecolor{cyan}{rgb}{0,.698,.698}
    \definecolor{lightgray}{gray}{0.5}
    
    % bright ansi colors
    \definecolor{darkgray}{gray}{0.25}
    \definecolor{lightred}{rgb}{1.0,0.39,0.28}
    \definecolor{lightgreen}{rgb}{0.48,0.99,0.0}
    \definecolor{lightblue}{rgb}{0.53,0.81,0.92}
    \definecolor{lightpurple}{rgb}{0.87,0.63,0.87}
    \definecolor{lightcyan}{rgb}{0.5,1.0,0.83}
    
    % commands and environments needed by pandoc snippets
    % extracted from the output of `pandoc -s`
    \DefineVerbatimEnvironment{Highlighting}{Verbatim}{commandchars=\\\{\}}
    % Add ',fontsize=\small' for more characters per line
    \newenvironment{Shaded}{}{}
    \newcommand{\KeywordTok}[1]{\textcolor[rgb]{0.00,0.44,0.13}{\textbf{{#1}}}}
    \newcommand{\DataTypeTok}[1]{\textcolor[rgb]{0.56,0.13,0.00}{{#1}}}
    \newcommand{\DecValTok}[1]{\textcolor[rgb]{0.25,0.63,0.44}{{#1}}}
    \newcommand{\BaseNTok}[1]{\textcolor[rgb]{0.25,0.63,0.44}{{#1}}}
    \newcommand{\FloatTok}[1]{\textcolor[rgb]{0.25,0.63,0.44}{{#1}}}
    \newcommand{\CharTok}[1]{\textcolor[rgb]{0.25,0.44,0.63}{{#1}}}
    \newcommand{\StringTok}[1]{\textcolor[rgb]{0.25,0.44,0.63}{{#1}}}
    \newcommand{\CommentTok}[1]{\textcolor[rgb]{0.38,0.63,0.69}{\textit{{#1}}}}
    \newcommand{\OtherTok}[1]{\textcolor[rgb]{0.00,0.44,0.13}{{#1}}}
    \newcommand{\AlertTok}[1]{\textcolor[rgb]{1.00,0.00,0.00}{\textbf{{#1}}}}
    \newcommand{\FunctionTok}[1]{\textcolor[rgb]{0.02,0.16,0.49}{{#1}}}
    \newcommand{\RegionMarkerTok}[1]{{#1}}
    \newcommand{\ErrorTok}[1]{\textcolor[rgb]{1.00,0.00,0.00}{\textbf{{#1}}}}
    \newcommand{\NormalTok}[1]{{#1}}
    
    % Define a nice break command that doesn't care if a line doesn't already
    % exist.
    \def\br{\hspace*{\fill} \\* }
    % Math Jax compatability definitions
    \def\gt{>}
    \def\lt{<}
    % Document parameters
    \title{hotHandEffect}
    
    
    

    % Pygments definitions
    
\makeatletter
\def\PY@reset{\let\PY@it=\relax \let\PY@bf=\relax%
    \let\PY@ul=\relax \let\PY@tc=\relax%
    \let\PY@bc=\relax \let\PY@ff=\relax}
\def\PY@tok#1{\csname PY@tok@#1\endcsname}
\def\PY@toks#1+{\ifx\relax#1\empty\else%
    \PY@tok{#1}\expandafter\PY@toks\fi}
\def\PY@do#1{\PY@bc{\PY@tc{\PY@ul{%
    \PY@it{\PY@bf{\PY@ff{#1}}}}}}}
\def\PY#1#2{\PY@reset\PY@toks#1+\relax+\PY@do{#2}}

\expandafter\def\csname PY@tok@gd\endcsname{\def\PY@tc##1{\textcolor[rgb]{0.63,0.00,0.00}{##1}}}
\expandafter\def\csname PY@tok@gu\endcsname{\let\PY@bf=\textbf\def\PY@tc##1{\textcolor[rgb]{0.50,0.00,0.50}{##1}}}
\expandafter\def\csname PY@tok@gt\endcsname{\def\PY@tc##1{\textcolor[rgb]{0.00,0.27,0.87}{##1}}}
\expandafter\def\csname PY@tok@gs\endcsname{\let\PY@bf=\textbf}
\expandafter\def\csname PY@tok@gr\endcsname{\def\PY@tc##1{\textcolor[rgb]{1.00,0.00,0.00}{##1}}}
\expandafter\def\csname PY@tok@cm\endcsname{\let\PY@it=\textit\def\PY@tc##1{\textcolor[rgb]{0.25,0.50,0.50}{##1}}}
\expandafter\def\csname PY@tok@vg\endcsname{\def\PY@tc##1{\textcolor[rgb]{0.10,0.09,0.49}{##1}}}
\expandafter\def\csname PY@tok@vi\endcsname{\def\PY@tc##1{\textcolor[rgb]{0.10,0.09,0.49}{##1}}}
\expandafter\def\csname PY@tok@vm\endcsname{\def\PY@tc##1{\textcolor[rgb]{0.10,0.09,0.49}{##1}}}
\expandafter\def\csname PY@tok@mh\endcsname{\def\PY@tc##1{\textcolor[rgb]{0.40,0.40,0.40}{##1}}}
\expandafter\def\csname PY@tok@cs\endcsname{\let\PY@it=\textit\def\PY@tc##1{\textcolor[rgb]{0.25,0.50,0.50}{##1}}}
\expandafter\def\csname PY@tok@ge\endcsname{\let\PY@it=\textit}
\expandafter\def\csname PY@tok@vc\endcsname{\def\PY@tc##1{\textcolor[rgb]{0.10,0.09,0.49}{##1}}}
\expandafter\def\csname PY@tok@il\endcsname{\def\PY@tc##1{\textcolor[rgb]{0.40,0.40,0.40}{##1}}}
\expandafter\def\csname PY@tok@go\endcsname{\def\PY@tc##1{\textcolor[rgb]{0.53,0.53,0.53}{##1}}}
\expandafter\def\csname PY@tok@cp\endcsname{\def\PY@tc##1{\textcolor[rgb]{0.74,0.48,0.00}{##1}}}
\expandafter\def\csname PY@tok@gi\endcsname{\def\PY@tc##1{\textcolor[rgb]{0.00,0.63,0.00}{##1}}}
\expandafter\def\csname PY@tok@gh\endcsname{\let\PY@bf=\textbf\def\PY@tc##1{\textcolor[rgb]{0.00,0.00,0.50}{##1}}}
\expandafter\def\csname PY@tok@ni\endcsname{\let\PY@bf=\textbf\def\PY@tc##1{\textcolor[rgb]{0.60,0.60,0.60}{##1}}}
\expandafter\def\csname PY@tok@nl\endcsname{\def\PY@tc##1{\textcolor[rgb]{0.63,0.63,0.00}{##1}}}
\expandafter\def\csname PY@tok@nn\endcsname{\let\PY@bf=\textbf\def\PY@tc##1{\textcolor[rgb]{0.00,0.00,1.00}{##1}}}
\expandafter\def\csname PY@tok@no\endcsname{\def\PY@tc##1{\textcolor[rgb]{0.53,0.00,0.00}{##1}}}
\expandafter\def\csname PY@tok@na\endcsname{\def\PY@tc##1{\textcolor[rgb]{0.49,0.56,0.16}{##1}}}
\expandafter\def\csname PY@tok@nb\endcsname{\def\PY@tc##1{\textcolor[rgb]{0.00,0.50,0.00}{##1}}}
\expandafter\def\csname PY@tok@nc\endcsname{\let\PY@bf=\textbf\def\PY@tc##1{\textcolor[rgb]{0.00,0.00,1.00}{##1}}}
\expandafter\def\csname PY@tok@nd\endcsname{\def\PY@tc##1{\textcolor[rgb]{0.67,0.13,1.00}{##1}}}
\expandafter\def\csname PY@tok@ne\endcsname{\let\PY@bf=\textbf\def\PY@tc##1{\textcolor[rgb]{0.82,0.25,0.23}{##1}}}
\expandafter\def\csname PY@tok@nf\endcsname{\def\PY@tc##1{\textcolor[rgb]{0.00,0.00,1.00}{##1}}}
\expandafter\def\csname PY@tok@si\endcsname{\let\PY@bf=\textbf\def\PY@tc##1{\textcolor[rgb]{0.73,0.40,0.53}{##1}}}
\expandafter\def\csname PY@tok@s2\endcsname{\def\PY@tc##1{\textcolor[rgb]{0.73,0.13,0.13}{##1}}}
\expandafter\def\csname PY@tok@nt\endcsname{\let\PY@bf=\textbf\def\PY@tc##1{\textcolor[rgb]{0.00,0.50,0.00}{##1}}}
\expandafter\def\csname PY@tok@nv\endcsname{\def\PY@tc##1{\textcolor[rgb]{0.10,0.09,0.49}{##1}}}
\expandafter\def\csname PY@tok@s1\endcsname{\def\PY@tc##1{\textcolor[rgb]{0.73,0.13,0.13}{##1}}}
\expandafter\def\csname PY@tok@dl\endcsname{\def\PY@tc##1{\textcolor[rgb]{0.73,0.13,0.13}{##1}}}
\expandafter\def\csname PY@tok@ch\endcsname{\let\PY@it=\textit\def\PY@tc##1{\textcolor[rgb]{0.25,0.50,0.50}{##1}}}
\expandafter\def\csname PY@tok@m\endcsname{\def\PY@tc##1{\textcolor[rgb]{0.40,0.40,0.40}{##1}}}
\expandafter\def\csname PY@tok@gp\endcsname{\let\PY@bf=\textbf\def\PY@tc##1{\textcolor[rgb]{0.00,0.00,0.50}{##1}}}
\expandafter\def\csname PY@tok@sh\endcsname{\def\PY@tc##1{\textcolor[rgb]{0.73,0.13,0.13}{##1}}}
\expandafter\def\csname PY@tok@ow\endcsname{\let\PY@bf=\textbf\def\PY@tc##1{\textcolor[rgb]{0.67,0.13,1.00}{##1}}}
\expandafter\def\csname PY@tok@sx\endcsname{\def\PY@tc##1{\textcolor[rgb]{0.00,0.50,0.00}{##1}}}
\expandafter\def\csname PY@tok@bp\endcsname{\def\PY@tc##1{\textcolor[rgb]{0.00,0.50,0.00}{##1}}}
\expandafter\def\csname PY@tok@c1\endcsname{\let\PY@it=\textit\def\PY@tc##1{\textcolor[rgb]{0.25,0.50,0.50}{##1}}}
\expandafter\def\csname PY@tok@fm\endcsname{\def\PY@tc##1{\textcolor[rgb]{0.00,0.00,1.00}{##1}}}
\expandafter\def\csname PY@tok@o\endcsname{\def\PY@tc##1{\textcolor[rgb]{0.40,0.40,0.40}{##1}}}
\expandafter\def\csname PY@tok@kc\endcsname{\let\PY@bf=\textbf\def\PY@tc##1{\textcolor[rgb]{0.00,0.50,0.00}{##1}}}
\expandafter\def\csname PY@tok@c\endcsname{\let\PY@it=\textit\def\PY@tc##1{\textcolor[rgb]{0.25,0.50,0.50}{##1}}}
\expandafter\def\csname PY@tok@mf\endcsname{\def\PY@tc##1{\textcolor[rgb]{0.40,0.40,0.40}{##1}}}
\expandafter\def\csname PY@tok@err\endcsname{\def\PY@bc##1{\setlength{\fboxsep}{0pt}\fcolorbox[rgb]{1.00,0.00,0.00}{1,1,1}{\strut ##1}}}
\expandafter\def\csname PY@tok@mb\endcsname{\def\PY@tc##1{\textcolor[rgb]{0.40,0.40,0.40}{##1}}}
\expandafter\def\csname PY@tok@ss\endcsname{\def\PY@tc##1{\textcolor[rgb]{0.10,0.09,0.49}{##1}}}
\expandafter\def\csname PY@tok@sr\endcsname{\def\PY@tc##1{\textcolor[rgb]{0.73,0.40,0.53}{##1}}}
\expandafter\def\csname PY@tok@mo\endcsname{\def\PY@tc##1{\textcolor[rgb]{0.40,0.40,0.40}{##1}}}
\expandafter\def\csname PY@tok@kd\endcsname{\let\PY@bf=\textbf\def\PY@tc##1{\textcolor[rgb]{0.00,0.50,0.00}{##1}}}
\expandafter\def\csname PY@tok@mi\endcsname{\def\PY@tc##1{\textcolor[rgb]{0.40,0.40,0.40}{##1}}}
\expandafter\def\csname PY@tok@kn\endcsname{\let\PY@bf=\textbf\def\PY@tc##1{\textcolor[rgb]{0.00,0.50,0.00}{##1}}}
\expandafter\def\csname PY@tok@cpf\endcsname{\let\PY@it=\textit\def\PY@tc##1{\textcolor[rgb]{0.25,0.50,0.50}{##1}}}
\expandafter\def\csname PY@tok@kr\endcsname{\let\PY@bf=\textbf\def\PY@tc##1{\textcolor[rgb]{0.00,0.50,0.00}{##1}}}
\expandafter\def\csname PY@tok@s\endcsname{\def\PY@tc##1{\textcolor[rgb]{0.73,0.13,0.13}{##1}}}
\expandafter\def\csname PY@tok@kp\endcsname{\def\PY@tc##1{\textcolor[rgb]{0.00,0.50,0.00}{##1}}}
\expandafter\def\csname PY@tok@w\endcsname{\def\PY@tc##1{\textcolor[rgb]{0.73,0.73,0.73}{##1}}}
\expandafter\def\csname PY@tok@kt\endcsname{\def\PY@tc##1{\textcolor[rgb]{0.69,0.00,0.25}{##1}}}
\expandafter\def\csname PY@tok@sc\endcsname{\def\PY@tc##1{\textcolor[rgb]{0.73,0.13,0.13}{##1}}}
\expandafter\def\csname PY@tok@sb\endcsname{\def\PY@tc##1{\textcolor[rgb]{0.73,0.13,0.13}{##1}}}
\expandafter\def\csname PY@tok@sa\endcsname{\def\PY@tc##1{\textcolor[rgb]{0.73,0.13,0.13}{##1}}}
\expandafter\def\csname PY@tok@k\endcsname{\let\PY@bf=\textbf\def\PY@tc##1{\textcolor[rgb]{0.00,0.50,0.00}{##1}}}
\expandafter\def\csname PY@tok@se\endcsname{\let\PY@bf=\textbf\def\PY@tc##1{\textcolor[rgb]{0.73,0.40,0.13}{##1}}}
\expandafter\def\csname PY@tok@sd\endcsname{\let\PY@it=\textit\def\PY@tc##1{\textcolor[rgb]{0.73,0.13,0.13}{##1}}}

\def\PYZbs{\char`\\}
\def\PYZus{\char`\_}
\def\PYZob{\char`\{}
\def\PYZcb{\char`\}}
\def\PYZca{\char`\^}
\def\PYZam{\char`\&}
\def\PYZlt{\char`\<}
\def\PYZgt{\char`\>}
\def\PYZsh{\char`\#}
\def\PYZpc{\char`\%}
\def\PYZdl{\char`\$}
\def\PYZhy{\char`\-}
\def\PYZsq{\char`\'}
\def\PYZdq{\char`\"}
\def\PYZti{\char`\~}
% for compatibility with earlier versions
\def\PYZat{@}
\def\PYZlb{[}
\def\PYZrb{]}
\makeatother


    % Exact colors from NB
    \definecolor{incolor}{rgb}{0.0, 0.0, 0.5}
    \definecolor{outcolor}{rgb}{0.545, 0.0, 0.0}



    
    % Prevent overflowing lines due to hard-to-break entities
    \sloppy 
    % Setup hyperref package
    \hypersetup{
      breaklinks=true,  % so long urls are correctly broken across lines
      colorlinks=true,
      urlcolor=blue,
      linkcolor=darkorange,
      citecolor=darkgreen,
      }
    % Slightly bigger margins than the latex defaults
    
    \geometry{verbose,tmargin=1in,bmargin=1in,lmargin=1in,rmargin=1in}
    
    

    \begin{document}
    
    
    \maketitle
    
    

    

    \section{Investigation of ``Hot-hand effect''}


    There is a widely held belief in basketball that some players have
periods of time where they are much better shooters than normal. This
can be called having a ``hot hand''. An example would be if a player hit
3 shots in a row, then many fans would expect that player to be more
likely (than their usual percentage) to hit their next shot.

Whether this is a real effect or some kind of cognitive bias has been
previously studied. For instance in 1984(83?, 85? FIX THIS),
{[}RESEARCHERS{]} looked at actual shooting results of the Philadelphia
76ers, free throws of the Boston Celtics, and controlled-experiment
shots of college students. Their data did not support the existence of a
hot-hand effect. This question has been revisited, for instance in
{[}PAPERS FROM 2005?? or so??{]} which also failed to find evidence to
support the existence of the hot-hand effect. Interestingly, in 2015
{[}RESEARCHERS{]} noted that the sampling method used in previous
studies was flawed. The flaw is subtle, but leads to some evidence for
the hot-hand effect.

My intent is to investigate recent shooting results of NBA players with
the goals of:

Looking for evidence to support (or reject) the existence of the
hot-hand effect.

Understand the subtleties of the sampling flaw found by
{[}RESEARCHERS{]}.


    \subsection{Getting the data}


    The reason I chose to study NBA data was that I found a resource that
makes shooting data easy to download for the NBA: www.nbasavant.com. I
have downloaded all shooting data for 2016-2017 and placed it in the
files nba\_savant\_\_.csv in the data/nba\_savant folder. {[}Note: you
can ostensibly download a .csv file of the shots data for the entire
year, but the files seem to be limited to 50,000 lines which is not
enough. That's why I split the data by month when downloading.{]}

There is some concern about the complete validity of the data. My
biggest concern is the data for April 2017 seems to be incomplete. There
are not near enough total shots for the month and spot checking some
players shows many fewer shots than expected for that player. However,
working with this data will at least provide a framework for studying
similar datasets.

Before reading in the data, we will load pandas, a python library used
for data analysis.

    \begin{Verbatim}[commandchars=\\\{\}]
{\color{incolor}In [{\color{incolor}1}]:} \PY{k+kn}{import} \PY{n+nn}{pandas} \PY{k+kn}{as} \PY{n+nn}{pd}
\end{Verbatim}

    Now, we can read in the data from the downloaded .csv files and store
the data as a DataFrame (a pandas data structure).

    \begin{Verbatim}[commandchars=\\\{\}]
{\color{incolor}In [{\color{incolor}2}]:} \PY{k+kn}{import} \PY{n+nn}{glob}
        \PY{n}{files\PYZus{}to\PYZus{}read} \PY{o}{=} \PY{n}{glob}\PY{o}{.}\PY{n}{glob}\PY{p}{(}\PY{l+s+s1}{\PYZsq{}}\PY{l+s+s1}{../data/nba\PYZus{}savant/*.csv}\PY{l+s+s1}{\PYZsq{}}\PY{p}{)}
        
        \PY{n}{shots} \PY{o}{=} \PY{n}{pd}\PY{o}{.}\PY{n}{concat}\PY{p}{(}\PY{p}{(}\PY{n}{pd}\PY{o}{.}\PY{n}{read\PYZus{}csv}\PY{p}{(}\PY{n}{f}\PY{p}{)} \PY{k}{for} \PY{n}{f} \PY{o+ow}{in} \PY{n}{files\PYZus{}to\PYZus{}read}\PY{p}{)}\PY{p}{)}
\end{Verbatim}

    We can look at the first few rows of data to get a sense of what data we
have (and if the read did what we expected it to do):

    \begin{Verbatim}[commandchars=\\\{\}]
{\color{incolor}In [{\color{incolor}3}]:} \PY{n}{shots}\PY{o}{.}\PY{n}{head}\PY{p}{(}\PY{p}{)}
\end{Verbatim}

            \begin{Verbatim}[commandchars=\\\{\}]
{\color{outcolor}Out[{\color{outcolor}3}]:}              name           team\_name   game\_date  season  espn\_player\_id  \textbackslash{}
        0  Andre Drummond     Detroit Pistons  2017-01-01    2016          6585.0   
        1    Nerlens Noel  Philadelphia 76ers  2017-01-30    2016       2991280.0   
        2       Jon Leuer     Detroit Pistons  2017-01-01    2016          6452.0   
        3   Dwight Howard       Atlanta Hawks  2017-01-29    2016          2384.0   
        4  Andre Drummond     Detroit Pistons  2017-01-01    2016          6585.0   
        
              team\_id  espn\_game\_id  period  minutes\_remaining  seconds\_remaining  \textbackslash{}
        0  1610612765     400899381       4                  5                 57   
        1  1610612755             0       3                  7                 18   
        2  1610612765     400899381       1                  3                 29   
        3  1610612737     400900132       4                  0                 45   
        4  1610612765     400899381       1                  7                 57   
        
              \ldots           shot\_type shot\_distance          opponent  x  y  dribbles  \textbackslash{}
        0     \ldots      2PT Field Goal             0        Miami Heat  0  1         0   
        1     \ldots      2PT Field Goal             0  Sacramento Kings  0  1         0   
        2     \ldots      2PT Field Goal             0        Miami Heat  0  1         0   
        3     \ldots      2PT Field Goal             0   New York Knicks  0  1         0   
        4     \ldots      2PT Field Goal             0        Miami Heat  0  1         0   
        
           touch\_time  defender\_name  defender\_distance  shot\_clock  
        0         0.0            NaN                0.0         0.0  
        1         0.0            NaN                0.0         0.0  
        2         0.0            NaN                0.0         0.0  
        3         0.0            NaN                0.0         0.0  
        4         0.0            NaN                0.0         0.0  
        
        [5 rows x 22 columns]
\end{Verbatim}
        
    Each row of the DataFrame consists of one shot (the observation) and 22
variables. Those variables are the columns. Note that the \ldots{}
indicates we are not seeing all of the columns. Pandas has a setting
that gives the maximum number of columns to print. It appears that value
defaults to 20. We can increase this value and then look at the first
few rows again (using new max of 60, but 22 would suffice).

    \begin{Verbatim}[commandchars=\\\{\}]
{\color{incolor}In [{\color{incolor}4}]:} \PY{n}{pd}\PY{o}{.}\PY{n}{set\PYZus{}option}\PY{p}{(}\PY{l+s+s1}{\PYZsq{}}\PY{l+s+s1}{display.max\PYZus{}columns}\PY{l+s+s1}{\PYZsq{}}\PY{p}{,} \PY{l+m+mi}{60}\PY{p}{)}
        \PY{n}{shots}\PY{o}{.}\PY{n}{head}\PY{p}{(}\PY{p}{)}
\end{Verbatim}

            \begin{Verbatim}[commandchars=\\\{\}]
{\color{outcolor}Out[{\color{outcolor}4}]:}              name           team\_name   game\_date  season  espn\_player\_id  \textbackslash{}
        0  Andre Drummond     Detroit Pistons  2017-01-01    2016          6585.0   
        1    Nerlens Noel  Philadelphia 76ers  2017-01-30    2016       2991280.0   
        2       Jon Leuer     Detroit Pistons  2017-01-01    2016          6452.0   
        3   Dwight Howard       Atlanta Hawks  2017-01-29    2016          2384.0   
        4  Andre Drummond     Detroit Pistons  2017-01-01    2016          6585.0   
        
              team\_id  espn\_game\_id  period  minutes\_remaining  seconds\_remaining  \textbackslash{}
        0  1610612765     400899381       4                  5                 57   
        1  1610612755             0       3                  7                 18   
        2  1610612765     400899381       1                  3                 29   
        3  1610612737     400900132       4                  0                 45   
        4  1610612765     400899381       1                  7                 57   
        
           shot\_made\_flag          action\_type       shot\_type  shot\_distance  \textbackslash{}
        0               1  Alley Oop Dunk Shot  2PT Field Goal              0   
        1               1  Alley Oop Dunk Shot  2PT Field Goal              0   
        2               0  Alley Oop Dunk Shot  2PT Field Goal              0   
        3               1  Alley Oop Dunk Shot  2PT Field Goal              0   
        4               1  Alley Oop Dunk Shot  2PT Field Goal              0   
        
                   opponent  x  y  dribbles  touch\_time  defender\_name  \textbackslash{}
        0        Miami Heat  0  1         0         0.0            NaN   
        1  Sacramento Kings  0  1         0         0.0            NaN   
        2        Miami Heat  0  1         0         0.0            NaN   
        3   New York Knicks  0  1         0         0.0            NaN   
        4        Miami Heat  0  1         0         0.0            NaN   
        
           defender\_distance  shot\_clock  
        0                0.0         0.0  
        1                0.0         0.0  
        2                0.0         0.0  
        3                0.0         0.0  
        4                0.0         0.0  
\end{Verbatim}
        

    \subsection{Strategy}


    For an initial investigation, I plan on making the following
assumptions:

Each player's shots are to be investigated as one sequence throughout
the entire year. A different, and perhaps more useful, choice would be
to split up each players shots by game. We will look at the data split
by game later.

To consider the existence of the hot-hand effect we will only look at
whether the previous shot was a make or miss. The literature standard
seems to be looking at shooting percentages after streaks of 1, 2, or 3
consecutive makes or misses. We will handle more complicated scenarios
later.

The only variables we will consider for analyzing each shot are those
that determine:

Which player took the shot (name, espn\_player\_id)

When the shot was taken (game\_date, period, minutes\_remaining,
seconds\_remaining) {[}Note: these are used to determine the order the
shots occurred in.{]}

Whether the shot was a make or miss (shot\_made\_flag)

We will not, at least initially, be considering other variables such as
those associated to shot difficulty, opponents, or effects of other
players shooting on a given night.


    \subsection{Rearranging the data}


    Let's start by removing columns we are not interested in. Actually, we
are only keeping the columns we are interested in.

    \begin{Verbatim}[commandchars=\\\{\}]
{\color{incolor}In [{\color{incolor}5}]:} \PY{n}{shots} \PY{o}{=} \PY{n}{shots}\PY{p}{[}\PY{p}{:}\PY{p}{]}\PY{p}{[}\PY{p}{[}\PY{l+s+s1}{\PYZsq{}}\PY{l+s+s1}{name}\PY{l+s+s1}{\PYZsq{}}\PY{p}{,} \PY{l+s+s1}{\PYZsq{}}\PY{l+s+s1}{game\PYZus{}date}\PY{l+s+s1}{\PYZsq{}}\PY{p}{,} \PY{l+s+s1}{\PYZsq{}}\PY{l+s+s1}{espn\PYZus{}player\PYZus{}id}\PY{l+s+s1}{\PYZsq{}}\PY{p}{,} \PY{l+s+s1}{\PYZsq{}}\PY{l+s+s1}{period}\PY{l+s+s1}{\PYZsq{}}\PY{p}{,} \PY{l+s+s1}{\PYZsq{}}\PY{l+s+s1}{minutes\PYZus{}remaining}\PY{l+s+s1}{\PYZsq{}}\PY{p}{,} \PY{l+s+s1}{\PYZsq{}}\PY{l+s+s1}{seconds\PYZus{}remaining}\PY{l+s+s1}{\PYZsq{}}\PY{p}{,} \PY{l+s+s1}{\PYZsq{}}\PY{l+s+s1}{shot\PYZus{}made\PYZus{}flag}\PY{l+s+s1}{\PYZsq{}}\PY{p}{]}\PY{p}{]}
        \PY{n}{shots}\PY{o}{.}\PY{n}{head}\PY{p}{(}\PY{p}{)}
\end{Verbatim}

            \begin{Verbatim}[commandchars=\\\{\}]
{\color{outcolor}Out[{\color{outcolor}5}]:}              name   game\_date  espn\_player\_id  period  minutes\_remaining  \textbackslash{}
        0  Andre Drummond  2017-01-01          6585.0       4                  5   
        1    Nerlens Noel  2017-01-30       2991280.0       3                  7   
        2       Jon Leuer  2017-01-01          6452.0       1                  3   
        3   Dwight Howard  2017-01-29          2384.0       4                  0   
        4  Andre Drummond  2017-01-01          6585.0       1                  7   
        
           seconds\_remaining  shot\_made\_flag  
        0                 57               1  
        1                 18               1  
        2                 29               0  
        3                 45               1  
        4                 57               1  
\end{Verbatim}
        
    Now we can sort the dataframe by date. That will help us to easily find
the previous shot for each player.

    \begin{Verbatim}[commandchars=\\\{\}]
{\color{incolor}In [{\color{incolor}6}]:} \PY{n}{shots} \PY{o}{=} \PY{n}{shots}\PY{o}{.}\PY{n}{sort\PYZus{}values}\PY{p}{(}\PY{n}{by}\PY{o}{=}\PY{p}{[}\PY{l+s+s1}{\PYZsq{}}\PY{l+s+s1}{game\PYZus{}date}\PY{l+s+s1}{\PYZsq{}}\PY{p}{,}\PY{l+s+s1}{\PYZsq{}}\PY{l+s+s1}{period}\PY{l+s+s1}{\PYZsq{}}\PY{p}{,}\PY{l+s+s1}{\PYZsq{}}\PY{l+s+s1}{minutes\PYZus{}remaining}\PY{l+s+s1}{\PYZsq{}}\PY{p}{,}\PY{l+s+s1}{\PYZsq{}}\PY{l+s+s1}{seconds\PYZus{}remaining}\PY{l+s+s1}{\PYZsq{}}\PY{p}{]}\PY{p}{,}\PY{n}{axis}\PY{o}{=}\PY{l+m+mi}{0}\PY{p}{,}\PY{n}{ascending}\PY{o}{=}\PY{p}{[}\PY{n+nb+bp}{True}\PY{p}{,}\PY{n+nb+bp}{True}\PY{p}{,}\PY{n+nb+bp}{False}\PY{p}{,}\PY{n+nb+bp}{False}\PY{p}{]}\PY{p}{)}
        \PY{n}{shots}\PY{o}{.}\PY{n}{head}\PY{p}{(}\PY{p}{)}
\end{Verbatim}

            \begin{Verbatim}[commandchars=\\\{\}]
{\color{outcolor}Out[{\color{outcolor}6}]:}                    name   game\_date  espn\_player\_id  period  \textbackslash{}
        6243        Rodney Hood  2016-10-25       2581177.0       1   
        675        Derrick Rose  2016-10-25          3456.0       1   
        3037  LaMarcus Aldridge  2016-10-25          2983.0       1   
        2133     AlFarouq Aminu  2016-10-25          4248.0       1   
        1792         Kevin Love  2016-10-25          3449.0       1   
        
              minutes\_remaining  seconds\_remaining  shot\_made\_flag  
        6243                 11                 44               1  
        675                  11                 40               1  
        3037                 11                 36               0  
        2133                 11                 27               1  
        1792                 11                 26               0  
\end{Verbatim}
        
    Now for the more interesting processing step. This will involve:

Group the shots by player.

Add a `previous\_shot\_made\_flag' for every shot that indicates if the
player made the previous shot.

Group and aggregate each player's data by `previous\_shot\_made\_flag'
to calculate the mean shooting percentage for both when the player
made/missed the previous shot.

Collect the individual player percentages on made/missed shot back into
a dataframe.

    \begin{Verbatim}[commandchars=\\\{\}]
{\color{incolor}In [{\color{incolor}7}]:} \PY{n}{shots\PYZus{}grouped\PYZus{}by\PYZus{}player} \PY{o}{=} \PY{n}{shots}\PY{o}{.}\PY{n}{groupby}\PY{p}{(}\PY{l+s+s1}{\PYZsq{}}\PY{l+s+s1}{espn\PYZus{}player\PYZus{}id}\PY{l+s+s1}{\PYZsq{}}\PY{p}{)}
        
        \PY{c+c1}{\PYZsh{} to collect the perecentages afer make/miss for each player}
        \PY{n}{player\PYZus{}shot\PYZus{}dict} \PY{o}{=} \PY{p}{\PYZob{}}\PY{p}{\PYZcb{}}
        
        \PY{k}{for} \PY{n+nb}{id}\PY{p}{,}\PY{n}{group} \PY{o+ow}{in} \PY{n}{shots\PYZus{}grouped\PYZus{}by\PYZus{}player}\PY{p}{:}
            \PY{n}{player\PYZus{}stats} \PY{o}{=} \PY{p}{\PYZob{}}\PY{p}{\PYZcb{}}
            \PY{n}{group}\PY{p}{[}\PY{l+s+s1}{\PYZsq{}}\PY{l+s+s1}{previous\PYZus{}shot\PYZus{}made\PYZus{}flag}\PY{l+s+s1}{\PYZsq{}}\PY{p}{]} \PY{o}{=} \PY{n}{group}\PY{p}{[}\PY{l+s+s1}{\PYZsq{}}\PY{l+s+s1}{shot\PYZus{}made\PYZus{}flag}\PY{l+s+s1}{\PYZsq{}}\PY{p}{]}\PY{o}{.}\PY{n}{shift}\PY{p}{(}\PY{l+m+mi}{1}\PY{p}{)}
            \PY{n}{player\PYZus{}stats}\PY{p}{[}\PY{n}{group}\PY{o}{.}\PY{n}{iloc}\PY{p}{[}\PY{l+m+mi}{0}\PY{p}{]}\PY{p}{[}\PY{l+s+s1}{\PYZsq{}}\PY{l+s+s1}{espn\PYZus{}player\PYZus{}id}\PY{l+s+s1}{\PYZsq{}}\PY{p}{]}\PY{p}{]} \PY{o}{=} \PY{n}{group}\PY{p}{[}\PY{p}{[}\PY{l+s+s1}{\PYZsq{}}\PY{l+s+s1}{shot\PYZus{}made\PYZus{}flag}\PY{l+s+s1}{\PYZsq{}}\PY{p}{,}\PY{l+s+s1}{\PYZsq{}}\PY{l+s+s1}{previous\PYZus{}shot\PYZus{}made\PYZus{}flag}\PY{l+s+s1}{\PYZsq{}}\PY{p}{]}\PY{p}{]}\PY{o}{.}\PY{n}{groupby}\PY{p}{(}\PY{l+s+s1}{\PYZsq{}}\PY{l+s+s1}{previous\PYZus{}shot\PYZus{}made\PYZus{}flag}\PY{l+s+s1}{\PYZsq{}}\PY{p}{)}\PY{o}{.}\PY{n}{aggregate}\PY{p}{(}\PY{p}{[}\PY{n}{numpy}\PY{o}{.}\PY{n}{mean}\PY{p}{,}\PY{n}{numpy}\PY{o}{.}\PY{n}{sum}\PY{p}{,}\PY{n}{numpy}\PY{o}{.}\PY{n}{count\PYZus{}nonzero}\PY{p}{]}\PY{p}{)}\PY{p}{[}\PY{l+s+s1}{\PYZsq{}}\PY{l+s+s1}{shot\PYZus{}made\PYZus{}flag}\PY{l+s+s1}{\PYZsq{}}\PY{p}{]}
            
            \PY{c+c1}{\PYZsh{} filter out players who did not attempt at least one shot after both a make and a miss}
            \PY{k}{if} \PY{o+ow}{not} \PY{n}{player\PYZus{}stats}\PY{p}{[}\PY{n+nb}{id}\PY{p}{]}\PY{o}{.}\PY{n}{empty} \PY{o+ow}{and} \PY{n}{player\PYZus{}stats}\PY{p}{[}\PY{n+nb}{id}\PY{p}{]}\PY{o}{.}\PY{n}{index}\PY{o}{.}\PY{n}{size} \PY{o}{==} \PY{l+m+mi}{2}\PY{p}{:}
                    \PY{n}{player\PYZus{}shot\PYZus{}dict}\PY{p}{[}\PY{n+nb}{id}\PY{p}{]} \PY{o}{=} \PY{p}{\PYZob{}}\PY{l+s+s1}{\PYZsq{}}\PY{l+s+s1}{percent\PYZus{}after\PYZus{}miss}\PY{l+s+s1}{\PYZsq{}}\PY{p}{:}\PY{n}{player\PYZus{}stats}\PY{p}{[}\PY{n+nb}{id}\PY{p}{]}\PY{o}{.}\PY{n}{loc}\PY{p}{[}\PY{l+m+mi}{0}\PY{p}{]}\PY{p}{[}\PY{l+s+s1}{\PYZsq{}}\PY{l+s+s1}{mean}\PY{l+s+s1}{\PYZsq{}}\PY{p}{]}\PY{p}{,} \PY{l+s+s1}{\PYZsq{}}\PY{l+s+s1}{percent\PYZus{}after\PYZus{}make}\PY{l+s+s1}{\PYZsq{}}\PY{p}{:}\PY{n}{player\PYZus{}stats}\PY{p}{[}\PY{n+nb}{id}\PY{p}{]}\PY{o}{.}\PY{n}{loc}\PY{p}{[}\PY{l+m+mi}{1}\PY{p}{]}\PY{p}{[}\PY{l+s+s1}{\PYZsq{}}\PY{l+s+s1}{mean}\PY{l+s+s1}{\PYZsq{}}\PY{p}{]}\PY{p}{\PYZcb{}}
        \PY{n}{player\PYZus{}shot\PYZus{}df} \PY{o}{=} \PY{n}{pd}\PY{o}{.}\PY{n}{DataFrame}\PY{o}{.}\PY{n}{from\PYZus{}dict}\PY{p}{(}\PY{n}{data}\PY{o}{=}\PY{n}{player\PYZus{}shot\PYZus{}dict}\PY{p}{,}\PY{n}{orient}\PY{o}{=}\PY{l+s+s1}{\PYZsq{}}\PY{l+s+s1}{index}\PY{l+s+s1}{\PYZsq{}}\PY{p}{)}
        \PY{n}{player\PYZus{}shot\PYZus{}df}\PY{o}{.}\PY{n}{head}\PY{p}{(}\PY{p}{)}
\end{Verbatim}

    \begin{Verbatim}[commandchars=\\\{\}]
-c:8: SettingWithCopyWarning: 
A value is trying to be set on a copy of a slice from a DataFrame.
Try using .loc[row\_indexer,col\_indexer] = value instead

See the caveats in the documentation: http://pandas.pydata.org/pandas-docs/stable/indexing.html\#indexing-view-versus-copy
    \end{Verbatim}

            \begin{Verbatim}[commandchars=\\\{\}]
{\color{outcolor}Out[{\color{outcolor}7}]:}        percent\_after\_miss  percent\_after\_make
        25.0             0.285714            0.000000
        136.0            0.380435            0.412429
        165.0            0.430435            0.386997
        272.0            0.385593            0.386667
        558.0            0.285714            0.500000
\end{Verbatim}
        
    We now have the shooting percentage for every player both after a miss
and after a make. If shooters do not get ``hot'', then it seems
reasonable that for each player that percent\_after\_miss and
percent\_after\_make should be roughly equal. If shooters do get
``hot'', then it seems reasonable that for each player that
percent\_after\_miss should typically be less than percent\_after\_make.


    \subsection{Analyzing the data}


    To start investigating these questions we can first plot the data in a
scatter plot. Points will be plotted with red points for a player that
has a higher percentage after a make and blue points for a player that
has a higher percentage after a miss.

    \begin{Verbatim}[commandchars=\\\{\}]
{\color{incolor}In [{\color{incolor}8}]:} \PY{n}{colors} \PY{o}{=} \PY{n}{np}\PY{o}{.}\PY{n}{where}\PY{p}{(}\PY{n}{player\PYZus{}shot\PYZus{}df}\PY{p}{[}\PY{l+s+s1}{\PYZsq{}}\PY{l+s+s1}{percent\PYZus{}after\PYZus{}miss}\PY{l+s+s1}{\PYZsq{}}\PY{p}{]}\PY{o}{\PYZgt{}}\PY{n}{player\PYZus{}shot\PYZus{}df}\PY{p}{[}\PY{l+s+s1}{\PYZsq{}}\PY{l+s+s1}{percent\PYZus{}after\PYZus{}make}\PY{l+s+s1}{\PYZsq{}}\PY{p}{]}\PY{p}{,} \PY{l+s+s1}{\PYZsq{}}\PY{l+s+s1}{b}\PY{l+s+s1}{\PYZsq{}}\PY{p}{,} \PY{l+s+s1}{\PYZsq{}}\PY{l+s+s1}{r}\PY{l+s+s1}{\PYZsq{}}\PY{p}{)}
        \PY{n}{player\PYZus{}shot\PYZus{}df}\PY{o}{.}\PY{n}{plot}\PY{p}{(}\PY{n}{x}\PY{o}{=}\PY{l+s+s1}{\PYZsq{}}\PY{l+s+s1}{percent\PYZus{}after\PYZus{}miss}\PY{l+s+s1}{\PYZsq{}}\PY{p}{,} \PY{n}{y}\PY{o}{=}\PY{l+s+s1}{\PYZsq{}}\PY{l+s+s1}{percent\PYZus{}after\PYZus{}make}\PY{l+s+s1}{\PYZsq{}}\PY{p}{,} \PY{n}{kind}\PY{o}{=}\PY{l+s+s1}{\PYZsq{}}\PY{l+s+s1}{scatter}\PY{l+s+s1}{\PYZsq{}}\PY{p}{,} \PY{n}{c}\PY{o}{=}\PY{n}{colors}\PY{p}{,} \PY{n}{figsize}\PY{o}{=}\PY{p}{(}\PY{l+m+mi}{12}\PY{p}{,}\PY{l+m+mi}{6}\PY{p}{)}\PY{p}{)}
\end{Verbatim}

            \begin{Verbatim}[commandchars=\\\{\}]
{\color{outcolor}Out[{\color{outcolor}8}]:} <matplotlib.axes.\_subplots.AxesSubplot at 0x7fb4b805c8d0>
\end{Verbatim}
        
    \begin{center}
    \adjustimage{max size={0.9\linewidth}{0.9\paperheight}}{hotHandEffect_files/hotHandEffect_23_1.png}
    \end{center}
    { \hspace*{\fill} \\}
    
    The previous scatter plot does not seem to allow us to defnitively
conclude much of anything. Players seem roughly split between those that
shoot better after a miss and those that shoot better after a make.

We can run statistical tests to evaluate the results objectively.

The first test we will try is a t-test. Our null hypothesis will be that
shooting percentage after a make and shooting percentage after a miss
are the same. The alternative hypothesis will be that shooting
percentage after a make is greater than shooting percentage after a
miss. Let's assume an alpha value of 0.05 which means that we need a
p-value less than 0.05 to reject the null hypothesis and conclude the
data support that players have a better shooting percentage after a
make. Note that this results in a one-tailed test. Annoyingly, python's
standard libraries seem to only compute p-values for a two-tailed test.
That means we need to divide the python calculated p-value by 2 and then
possibly subtracting from 1 (depending on whether the t-statistic is
positive or negative) to get our true p-value.

    \begin{Verbatim}[commandchars=\\\{\}]
{\color{incolor}In [{\color{incolor}9}]:} \PY{k+kn}{import} \PY{n+nn}{scipy.stats}
        
        \PY{p}{(}\PY{n}{t}\PY{p}{,} \PY{n}{p}\PY{p}{)} \PY{o}{=} \PY{n}{scipy}\PY{o}{.}\PY{n}{stats}\PY{o}{.}\PY{n}{ttest\PYZus{}1samp}\PY{p}{(}\PY{n}{player\PYZus{}shot\PYZus{}df}\PY{o}{.}\PY{n}{percent\PYZus{}after\PYZus{}make}\PY{o}{\PYZhy{}}\PY{n}{player\PYZus{}shot\PYZus{}df}\PY{o}{.}\PY{n}{percent\PYZus{}after\PYZus{}miss}\PY{p}{,}\PY{l+m+mi}{0}\PY{p}{)}
        \PY{k}{if} \PY{n}{t} \PY{o}{\PYZgt{}} \PY{l+m+mi}{0}\PY{p}{:}
            \PY{c+c1}{\PYZsh{} t\PYZgt{}0 implies percent\PYZus{}after\PYZus{}make is generally greater than percent\PYZus{}after\PYZus{}miss}
            \PY{c+c1}{\PYZsh{} so this is the tail that (at least somewhat) supports our alternative hypothesis}
            \PY{n}{p} \PY{o}{=} \PY{n}{p}\PY{o}{/}\PY{l+m+mi}{2}
        \PY{k}{else}\PY{p}{:}
            \PY{c+c1}{\PYZsh{} t\PYZlt{}0 implies percent\PYZus{}after\PYZus{}make is generally less than percent\PYZus{}after\PYZus{}miss}
            \PY{c+c1}{\PYZsh{} so this is the tail that definitely does not support our alternative hypothesis}
            \PY{n}{p} \PY{o}{=} \PY{l+m+mi}{1} \PY{o}{\PYZhy{}} \PY{n}{p}\PY{o}{/}\PY{l+m+mi}{2}
        \PY{n}{p}
\end{Verbatim}

            \begin{Verbatim}[commandchars=\\\{\}]
{\color{outcolor}Out[{\color{outcolor}9}]:} 0.99951561688842272
\end{Verbatim}
        
    So, our p-value is (much) larger than 0.05 which means the data do not
support the hot hand hypothesis. In fact, the data would have supported
the hypothesis that the average shooting percentage of players is higher
after a miss than after a make.

Where to go from here? There are a number of things we could tidy up.
Some of these include the following.

Each player's shots are viewed as one sequence for the entire year. We
could split each player's shots on a per game basis.

The t-test assumes that the distribution of differences between percent
after make and miss is normally distributed. We could look into the
validity of that assumption.

There seem to be a fair number of outliers in the scatter plot. It seems
likely that a lot of the outliers are players that took very few shots.
One option to deal with the outliers would be to only consider players
that took at least a certain number of shots. That seems tempting at
first, but once we split each players shots into individual games we
will need to consider relatively small numbers of shots anyway.
Hopefully, we can use appropriate statistics to account for a player
with a small number of shots. The essential problem we have is that we
are using something roughly akin to an average-of-averages which is
problematic with each inner average having a different sample size.

We can extend our analysis to look at more than the previous shot. Maybe
the previous n shots for some n=2 or n=3.

We could account for the different shot types. Maybe do something like
only looking at 3-point shots or jump shots.

We could apply machine learning (or other) techniques to look for
patterns in the players that do have a better shooting percentage after
a make or a miss.

We could investigate the grouping step of the data arranging. This step
takes a fair amount of time. Perhaps there is a more efficient way to
perform the same task.


    % Add a bibliography block to the postdoc
    
    
    
    \end{document}
